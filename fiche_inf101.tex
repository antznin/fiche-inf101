\documentclass[a4paper, 8pt]{article}

\usepackage[utf8]{inputenc}
\usepackage[francais]{babel}
\usepackage[T1]{fontenc}
\usepackage{fancyhdr}
\usepackage{amsmath}
\usepackage{framed}
\usepackage[margin=0.7cm]{geometry}
\usepackage{varwidth}
\usepackage{multicol}
\usepackage{amsfonts}
\usepackage{amssymb}
\usepackage[babel=true]{csquotes}
\usepackage{titlesec}
\usepackage{color}
\usepackage{fancybox}
 \usepackage[usenames,dvipsnames,pdf]{pstricks}
 \usepackage{epsfig}
 \usepackage{pst-grad} % For gradients
 \usepackage{pst-plot} % For axes
 \usepackage[space]{grffile} % For spaces in paths
 \usepackage{etoolbox} % For spaces in paths
\usepackage[vlined, boxruled]{algorithm2e}
\usepackage{pgf,tikz}
\usepackage{pgfplots}
\usetikzlibrary{arrows}
\tikzset{
  font={\fontsize{7pt}{12}\selectfont}}
\usetikzlibrary{graphs}

\def\changemargin#1#2{\list{}{\rightmargin#2\leftmargin#1}\item[]}
\let\endchangemargin=\endlist 

\titleformat{\section}
 {\sffamily\large\bfseries\raggedleft}{\thesection}{1em}{}[{\titlerule[1pt]}]
 
 
\titleformat{\subsection}
 {\sffamily\normalsize\bfseries\centering}{\thesection}{0em}{}[{\titlerule[0pt]}]


\title{\sffamily\textbf{Fiche de révision \\ INF101} \thanks{Cours suivi avec Olivier \bsc{Hudry} }}
\author{Antonin \bsc{Godard}\\Sylvain \bsc{Rager}}
\date{}

\begin{document}

\begin{multicols*}{3}
\setlength{\parindent}{0pt}
\maketitle

\begin{changemargin}{0.5cm}{0.5cm} 
{\footnotesize Ce document est à lire après avoir suivi le cours. Le document fait donc l'hypothèse que le lecteur connaît au préalable les notions dont le document parle.}
\end{changemargin}


\textbf{Complexité:} mesure du temps nécessaire de calcul pour finir l'algorithme.\\ Soit $(P)$ un problème. Soit $I$ une instance de $(P)$. Soit $A$ un algorithme pour $(P)$.\\ Soit $f_A(I)$ le nombre d'opérations élémentaires effectuées pour compléter $I$. On définit alors $c_A$ par :
\[
\begin{array}{cc|ccl}
c_A & : & \mathbb{N} & \to & \mathbb{N} \\
 & & n & \mapsto & \max(f_A(I)) \\
\end{array}
\]

\section*{Structures de Données}

\textbf{Variables:} emplacement précis en mémoire. Possède une adresse et une taille variable.

\textbf{Tableau:} collection de variables. Contient des cases numérotées. Caractérisé par :\begin{itemize}
\item son adresse
\item sa taille
\item nature des entités conservées (de même type pour un tableau).
\end{itemize}
On accède à l'adresse via :\\ $ T\left[ i\right] = T\left[ 0\right] + i\times t $. ($t$ : taille).

\textbf{Listes chaînées :} suite de données homogènes (même type) auxquelles on accède de proche en proche, chaque donnée contenant l'adresse de l'élément suivant. Avantage : pas de problème de dimensionnement. Inconvénient : plus d'accès direct.

\textbf{Pile :} collection ordonnée de données respectant la stratégie \enquote{Last in, first out} (LIFO).

\textbf{File :} De même, \enquote{First in, first out} (FIFO).

\textbf{Arbres :} Un arbre $A$ est de la forme $A = (R, \underbrace{A_1, \ldots, A_k}_{\text{Arbres}})$. $R$ est la racine l'arbre $A$.\\
Terminologie : Soit $N$ un nœud de $A$.
\begin{itemize}
\item Les fils de $N$ sont les racines des sous-arbres de $N$;
\item Un nœud interne est un nœud admettant au moins 1 fils;
\item une feuille est un nœud n'ayant pas de fils;
\item tout nœud autre que la racine admet un père unique.
\end{itemize}

Profondeur de $N$ dans $A$ :
\[
\begin{array}{rc|l}
\text{prof}(N) & = & 0 \textit{ si } N=R  \\
 & & \text{prof}(\text{père de }N) \textit{ sinon.}  \\
\end{array}
\]

Hauteur de $A$ : $h = \displaystyle \max_{N\in A}\left(\text{prof}(N)\right).$

\textbf{Arbre complet :} Chaque nœud a deux fils.

\textbf{Arbre parfait :} $\forall i$, $(0 \leqslant i < h)$, le niveau $i$ est saturé.

\section*{Tri}

\newcommand{\GO}{\text{O}}

\textbf{Problème :} On dispose d'une collection de $n$ éléments, on souhaite les ordonner sous forme croissante.

\textbf{Tri insertion :} \`{A} la $i$\ieme étape, on considère les $i$ premier éléments triés, l'élément $i+1$ est alors inséré dans les éléments déjà triés par comparaison.
\begin{algorithm}[H]
 \KwData{$T$ tableau d'indices 1 à $n$}
 \For{$i$ variant de 2 à $n$}{
  $j \longleftarrow i$ \;
  $cle \longleftarrow T\left[j\right]$ \;
  \While{$j \geqslant 2$ et $T\left[j\right] > cle$}{
  	$T\left[j\right] \longleftarrow T\left[j-1\right]$ \;
  	$j \longleftarrow j-1$ \;
  }
  $T\left[j\right] \longleftarrow cle$ \;
 }
 \caption{Tri Insertion $\GO(n^2)$}
\end{algorithm}

\textbf{Tri sélection :} On cherche le plus petit élément, on le met à la 1\iere position; on cherche le 2\ieme plus petit, et etc$\hdots$

\begin{algorithm}[H]
 \KwData{$T$ tableau d'indices 1 à $n$}
 \For{$i$ variant de 1 à $n-1$}{
  $indicePetit \longleftarrow i$ \;
  $min \longleftarrow T\left[i\right]$ \;
  \For{$j$ variant de 1 à $n+1$}{
  \If{$T\left[j\right] < min$}{
  		$indicePetit \longleftarrow j$ \;
  		$min \longleftarrow T\left[j\right]$ \;
  	}
  	échanger($T$, $i$, $indicePetit$) \;
  }
 }
 \caption{Tri Sélection $\GO(n^2)$}
\end{algorithm}

\textbf{Tri rapide :} On utilise une fonction nommée $partition$ sur une liste qui sélectionne une donnée $cle$, et qui range les données plus petites que $cle$ à gauche de $cle$, puis range à droite de $cle$ les éléments plus grand que $cle$. Il est défini récursivement la plupart du temps.
\begin{algorithm}[H]
 \KwData{Données du tableau $T$ entre les indices $g$ et $d$}
  $cle \longleftarrow T\left[g\right]$ \;
  $i \longleftarrow g+1$ \;
  $j \longleftarrow d$ \;
  \While{$j \geqslant i$}{
  	\While{$i \leqslant j$ et $T\left[i\right] \leqslant cle$}{$i \longleftarrow i+1$\;}
  	\While{$T\left[j\right] > cle$}{$j \longleftarrow j-1$\;}
  	\If{$i < j$}{
  		échanger($T$,$i$,$j$) \;
  		$i \longleftarrow i+1$\;
  		$j \longleftarrow j-1$\;
  	}
  }
  	échanger($T$,$g$,$j$)\;
  	\Return{j}
 \caption{partition($g$,$d$)}
\end{algorithm}
\begin{algorithm}[H]
 \KwData{$j$}
 \If{$g \leqslant d $}{
 	$j \longleftarrow$ partition($g$,$d$)\;
 	tri\_rapide($g$,$j-1$)\;
 	tri\_rapide($j+1$,$d$)\;
 }
 \caption{tri\_rapide($g$,$d$) $\GO(n\log n)$}
\end{algorithm}
\medskip

\textbf{Tri par arbre binaire de recherche (ABR) :} arbre binaire vérifiant :
\begin{changemargin}{0.5cm}{0.5cm} 
\begin{center}
$\forall N\in A$, la clé de $N$ est $\geqslant$ aux clés du sous-arbre gauche de $N$ et $\leqslant$ aux clés du sous-arbre droit de $N$.
\end{center}
\end{changemargin}

Le tri ABR se décompose en deux étapes :
\begin{enumerate}
\item Construction de l'ABR 
\item Parcours de l'ABR
\end{enumerate}

Un parcours infixe trie les clés de l'ABR, le tout en $\GO(n^2)$.

\medskip
\textbf{Tri tas :} on définit un tas comme un arbre dont les clés vérifient :
\begin{changemargin}{0.5cm}{0.5cm} 
\begin{center}
$\forall N\in A$, la clé de $N$ est $\geqslant$ à la clé des fils de $N$.
\end{center}
\end{changemargin}

Le tri tas se décompose en deux étapes :
\begin{enumerate}
\item construction du tas en $\GO(n\log(n))$
\item exploitation du tas en $\GO(n\log(n))$
\end{enumerate}

D'où la complexité en $\GO(n\log(n))$.

\begin{algorithm}[H]
 \KwData{$i$ est un entier et $cle$ est du même type que les données du tas}
  $i \longleftarrow p$ \;
  $cle \longleftarrow T\left[p\right]$ \;
  \While{$i \geqslant 2$ \text{et} $cle \geqslant T\left[\lfloor\frac{i}{2}\rfloor\right]$}{
  	$T\left[i\right]  \longleftarrow T\left[\lfloor\frac{i}{2}\rfloor\right]$ \;
  	$i \longleftarrow \lfloor\frac{i}{2}\rfloor$ \;
  	}
  	$T\left[i\right] \longleftarrow cle$ \;
 \caption{montée$(p)$}
\end{algorithm}

Principe du tri : on échange la racine avec la dernière feuille du tas. L'ex-racine est à la fin du tableau, on n'y touche plus. On va alors reconstruire le tas avec les données restantes en faisant descendre la donnée échangée.

\newcommand{\ig}{indice\_ grand}
\begin{algorithm}[H]
 \KwData{$p$ est le nombre de données et la donnée d'indice $q$ est diminuée}
  $trouve \longleftarrow FAUX$ \;
  $i \longleftarrow q$ \;
  $cle \longleftarrow T\left[q\right]$ \;
  \While{not $trouve$ \text{et} $2i\leqslant p$}
  {
  	\eIf{$2i==p$}
  	{$\ig \longleftarrow p$}{
  		\If{$T\left[2i\right] \geqslant T\left[2i+1\right]$}
  		{
  			$\ig \longleftarrow 2i+1$~\;
  		}
  	}
  	\eIf{$cle < T\left[\ig\right]$}
  	{
  		$T\left[i\right] \longleftarrow T\left[ \ig \right]$ \;
  		$i \longleftarrow \ig$ \;
  	}{
  	$trouve \longleftarrow VRAI$ \;
  	}
  }
  $T\left[i\right] \longleftarrow cle$ \;
 \caption{descente$(q,p)$}
\end{algorithm}

\begin{algorithm}[H]
 \KwData{$p$ est un entier}
  \For{$2\leqslant p\leqslant n$, $p\leftarrow p+1$}
  {
  	montée$(p)$ \;
  }
  \For{$n\geqslant p\geqslant 2$, $p\leftarrow p-1$}
  {
  	échanger$(T,1,p)$ \;
  	descente$(1,p-1)$ \;
  }
 \caption{Tri tas $\GO(n\log(n))$}
\end{algorithm}

\section*{Fonctions de hachage}

Pour gérer une collision en hachage, on peut faire appel à un  \enquote{hachage linéaire}. Si, lors de la construction de la table, on veut ranger un mot mais que la place est déjà occupée, alors on se déplace cycliquement vers la droite (càd qu'on repart du début du tableau à chaque fois), puis on range le mot dans une case libre.

\section*{Codage de \bsc{Huffman}}

\textbf{Problème :} on souhaite envoyer un texte en l'envoyant en binaire afin d'optimiser la longueur du texte codé.

\newcommand{\occ}{\text{occ}}
\newcommand{\lng}{\text{l}}
On définit :
\begin{itemize}
\item $\occ(x) =$ nombre d'occurence de $x$
\item $\lng(x) =$ longueur de $x$
\end{itemize}

Le codage respecte la \textbf{règle du préfixe} : toute codage de caractère ne peut être le préfixe du codage d'un autre caractère. On associe un arbre binaire à ce codage dont les feuilles correspondent au caractère et chaque branche gauche représente un 0 et chaque branche droite représente un 1.

Long. du texte :
$L = \sum_{x}{\occ(x)\times p(x)}$

\textbf{Propositions :}
\begin{enumerate}
\item tout arbre optimal est \emph{complet}
\item $\exists$ un arbre optimal dans lequel les 2 caractères les moins fréquents ($\alpha$ \& $\beta$) sont associés aux feuilles de profondeur maximum
\item Idem, avec $\alpha$ et $\beta$ de même père, donc frères
\end{enumerate}

%\begin{center}
%\begin{tikzpicture}
%\tikzset{
%  node/.style = {circle,
%                     draw}
%}
%\node {root}
%[level distance=7mm,
%level 1/.style={sibling distance=20mm},
%level 2/.style={sibling distance=10mm},
%level 3/.style={sibling distance=5mm}]
%child {node {left}
%	child {node {child}}
%	child {node {child}}
%	}
%child {node {right}
%	child {node {child}}
%	child {node {child}}
%};
%\end{tikzpicture}
%\end{center}

\section*{Théorie des graphes}

\newcommand{\GXA}{$G = (X,A)$ }

\subsection*{Graphes Non-Orientés}
Un graphe non-orienté \GXA est un couple d'ensembles finis avec :
\begin{itemize}
\item $X = \left\{\text{sommets}\right\}$
\item $A = \left\{\mathbf{ar\hat{e}tes}\right\}$, les arêtes étant de la forme $\{x,y\}$ (le sens compte).
\item $m = \Omega(X)$ (cardinal)
\item $n = \Omega(Y)$
\item Pour une arête $a = \{x,y\}$ :
	\begin{itemize}
	\item $a$ est adjacente à $x$ et $y$
	\item $x$ et $y$ sont les extrémités de $a$
	\item $x$ et $y$ sont voisins
	\end{itemize}
\item Voisinage de $x$\\ $V(x) = \{y \in X / \{x,y\} \in A$
\item $d(x) = \Omega(V(x))$ et \begin{flushright}
$\sum_{x\in X}{d(x)} = 2m$
\end{flushright}
\item Chaîne : suite de sommets $x_1,x_2,\hdots,x_k$ avec :
\begin{enumerate}
\item $\forall \, 1\leqslant i \leqslant k \quad  x_i \in X$
\item $\forall \, 1\leqslant i \leqslant k-1 \quad \{x_i,x_{i+1}\} \in A$
\item $\forall \, 1< i< k \quad x_{i-1} \neq x_{i+1}$
\end{enumerate}
\item Chaîne simple (resp. élémentaire) : ne passe pas $2\times$ par une même arête (resp. sommet)
\item $x_1,x_2,\hdots,x_k$ élémentaire $\Longleftrightarrow \quad \forall \, 1\leqslant i,j \leqslant k \quad  x_i \neq x_j$
\item Cycle : suite de sommets $x_1,x_2,\hdots,x_k,(x_1)$ avec :
\begin{enumerate}
\item $\forall \, 1\leqslant i \leqslant k \quad  x_i \in X$
\item $\forall \, 1\leqslant i \leqslant k-1 \quad \{x_i,x_{i+1}\} \in A \,et\, \{x_k,x_{1}\} \in A$
\item $\forall \, 1< i< k \quad x_{i-1} \neq x_{i+1}$ \\$ et\, x_2 \neq x_k \,et\, x_1 \neq x_{k-1}$
\end{enumerate}
\item $G$ est dit connexe si $\forall\,x,y\in X \quad \exists\,\text{chaîne entre}\,x\,\text{et}\,y$. On définit de même les composantes connexes qui sont les sous-graphes connexes de $G$.
\item Une arête de $G$ est un \textit{isthme} si la suppression de cette arête fait croître le nombre de composantes connexes de $G$.
\item Un arbre est un graphe connexe sans cycle.
\end{itemize}

Exemple :
\begin{center}
\begin{tikzpicture}[scale = 1.5]

\tikzset{vertex/.style = {inner sep=2pt, draw = none}}
\tikzset{edge/.style = {-}}
% vertices

\node[vertex] (a) at  (0,.5) {$a$};
\node[vertex] (b) at  (-.5,0) {$b$};
\node[vertex] (c) at  (.5,1) {$c$};
\node[vertex] (d) at  (1,0) {$d$};
\node[vertex] (e) at  (1.7,1) {$e$};
\node[vertex] (f) at  (-1,1) {$f$};


%edges
\draw[edge] (a) to node[above]{7} (b);
\draw[edge] (a) to node[left]{3} (c);
\draw[edge] (f) to node[above]{4}(c);
\draw[edge] (e) to node[above]{0} (c);
\draw[edge] (d) to node[right]{-5} (e);
\draw[edge] (b) to node[below]{3} (d);
\draw[edge] (b) to node[left]{-1} (f);
\draw[edge] (d) to node[left]{-2} (c);
\end{tikzpicture}
\end{center}

Codage d'un graphe :

\begin{itemize}
	\item Matrice d'adjacence symétrique où a est la 1\iere colonne ou 1\iere ligne. Ici : \\
	 \[M = \left(\begin{smallmatrix}
	+\infty & 7 & 3 & +\infty & +\infty & +\infty \\ 
	 & +\infty & +\infty & 3 & +\infty & -1 \\ 
	 &  & +\infty & -2 & 0 & 4 \\ 
	 &  &  & +\infty & -5 & +\infty \\ 
	 & (sym) &  &  & +\infty & +\infty \\ 
	 &  &  &  &  & +\infty
	\end{smallmatrix} \right)\]
	
	\item Liste d'adjacence, où chaque ligne décrit les extrémités triées d'un élément avec leurs poids. Ici :
	\[\begin{smallmatrix}
	 a & \longrightarrow & b,7 & \longrightarrow & c,3 &  & &  &  \\ 
	 b & \longrightarrow & a,7 & \longrightarrow & d,3 & \longrightarrow & f,-1 &  &  \\ 
	 c & \longrightarrow & a,3 & \longrightarrow & d,-2 & \longrightarrow & e,0 & \longrightarrow & f,4 \\ 
	 d & \longrightarrow & b,3 & \longrightarrow & c,-2 & \longrightarrow & e,-5 &  &  \\ 
	 e & \longrightarrow & c,0 & \longrightarrow & d,-5 &  &  &  &  \\ 
	 f & \longrightarrow & b,-1 & \longrightarrow & c,4 &  &  &  & 
	 \end{smallmatrix}  \]
\end{itemize}

\subsection*{Graphe Orientés}
Un graphe orienté \GXA est un couple d'ensembles finis avec :
\begin{itemize}
\item $X = \left\{\text{sommets}\right\}$
\item $A = \left\{\mathbf{arcs}\right\}$, les arcs étant de la forme $(x,y)$ (le sens compte).
\item $m = \Omega(X)$ (cardinal)
\item $n = \Omega(Y)$
\item Pour un arc $(x,y)$ :
	\begin{itemize}
	\item $x$ est prédécesseur de $y$
	\item $y$ est successeur de $x$
	\item $V^+(x) =$ voisinage extérieur de $x$
	\item $V^-(x) =$ voisinage intérieur de $x$
	\end{itemize}
\end{itemize}
\medskip
Exemple :
\begin{center}
\begin{tikzpicture}[scale = 1.5]

\tikzset{vertex/.style = {inner sep=2pt, draw = none}}
\tikzset{edge/.style = {->,> = latex'}}
% vertices
\node[vertex] (a) at  (0,2) {$A$};
\node[vertex] (e) at  (-1,1) {$E$};
\node[vertex] (b) at  (1,1) {$B$};
\node[vertex] (d) at  (-1,0) {$D$};
\node[vertex] (c) at  (1,0) {$C$};

%edges
\draw[edge] (a) to node [left] {5} (e);
\draw[edge] (b) to node [right] {2} (c);
\draw[edge] (c) to node [above] {-3} (d);

\draw[edge] (a)  to[bend left = 15] node [right] {1} (b);
\draw[edge] (b) to[bend left = 15] node [left] {7} (a);

\draw[edge] (b) to[bend left = 15] node [below] {6} (e);
\draw[edge] (e) to[bend left = 15] node [above] {-4} (b);

\end{tikzpicture}
\end{center}

On définit alors de manière analogue la matrice d'adjacence de $G$. La liste d'adjacence distingue la liste des successeurs et la liste des prédécesseurs. 

\subsection*{Arbre couvant de poids minimum}

Étant donné un graphe $G$ non-orienté et pondéré, on cherche un arbre couvant de poids minimum sur ce graphe. 

On utilise l'algorithme de \textsc{Krusgal} qui consiste, pour un graphe \emph{connexe} de $n$ sommets, à sélectionner à chaque itération une nouvelle arête du graphe qui ne crée pas de cycle. Il se déroule en deux étapes :
\begin{enumerate}
\item trier les arêtes par ordre croissant
\item tant que l'on a pas retenu $n-1$ arêtes, on procède de la sorte : considérer, dans l'ordre du tri, la première arête non examinée, puis appliquer la condition du paragraphe ci-dessus
\end{enumerate}

\newcommand{\CC}{\text{CC}}
\begin{algorithm}[H]
\KwData{$T$ l'arbre couvant de poids minimum}
 Trier les arêtes $\longrightarrow L$ \;
 \For{$i$ de 1 à $n$}{
 	$\CC(i) \longleftarrow i$ \;
 }
 $ComptL \longleftarrow 1$ \;
 $ComptT \longleftarrow 0$ \;
 \While{$CompteurT < n-1$}{
 	Soit $\{x,y\}$ l'arête $L\left[ComptL\right]$\;
 	$ComptL \longleftarrow ComptL + 1$\;
 	\If{$\CC(x) \neq \CC(y)$}{
 		$ComptT \longleftarrow ComptT +1$\;
 		$T(ComptT) \longleftarrow \{x,y\}$\;
 		$aux \longleftarrow \CC(y)$\;
 		\For{$i$ de 1 à $n$}{
 			\If{$\CC(i) = aux$}
 			{
 				$\CC(i) \leftarrow \CC(x)$\;
 			}
 		}
 	}
 }
 
 \caption{Algo. de Krusgal $\GO(n^2+m\log m)$}
\end{algorithm}

\section*{Plus courts chemins}

\subsection*{Cas des graphes sans circuits : algorithme de Bellman}

\newcommand{\pere}{\text{père}}
\newcommand{\Vm}{V^-}
\newcommand{\Vp}{V^+}
\newcommand{\tred}[1]{{\color{red}#1}}

Soit \GXA orienté pondéré par $p\,:\,A\,\rightarrow\,\mathbb{R}$ \emph{sans-circuits}, et un sommet $r\in X$.

\textbf{Problème :} déterminer, pour $x\neq r$, un $\underset{\tred{plch.}}{pcch.}$ de $r$ à $x$.

On définit deux attributs :
\begin{itemize}
\item $\pi(x) =$ poids d'un pcch. de $r$ à $x$
\item $\pere(x) =$ prédécesseur de $x$ sur un tel chemin
\end{itemize}

\[ \pi(x) = \displaystyle \min_{y\in \Vm(x)}\left[\pi(y) + p(y,x)\right] \]

Une \textbf{numération topologique} de $G$ orienté est une bijection $\varphi\,:\,X\rightarrow\,\{1,2,\ldots,n\}$ tq. $\forall (x,y)\in A, \,\varphi(x) < \varphi(y)$.

%inclure exemple

\begin{algorithm}[H]
 Déterminer une numérotation topologique $\varphi$ et $\varphi^{-1}$ \;
 Pour $x\neq r$, $\pi(x) \longleftarrow \underset{\tred{-\infty}}{+\infty}$\;
 $\pi(r) \longleftarrow 0$ \;
 
 \For{$i$ variant de 2 à $n$}
 {
 	$x \longleftarrow \varphi^{-1}(i)$ \;
 	$\pi(x) \longleftarrow \min_{y\in \Vm(x)}\left[\pi(y) + p(y,x)\right]$ \;
 	{\footnotesize \tred{$\pi(x) \leftarrow\max_{y\in \Vm(x)}\left[\pi(y) + p(y,x)\right]$\;} }
 	$\pere(x) \longleftarrow y^*\in \Vm(x)$ avec $\pi(x) = \pi(y^*) + p(y^*,x)$ \;
 }
 
 \caption{Algorithme de \textsc{Bellman}}
\end{algorithm}

 \subsection*{Problème de l'ordonnancement}
 
\newcommand{\Dmin}{D_{\text{min}}} 
\newcommand{\debut}{\text{début}}
\newcommand{\fin}{\text{fin}}
 
 Soit $T$ un ensemble de $p$ tâches $t_i$ ($1\leqslant i\leqslant p$). Chaque tâche $t_i$ possède une durée $d_i$. On doit respecter des contraintes de la forme : avant de commencer $x$, on doit avoir fini $y$. On peut réaliser les tâches en parallèle.
 
De quelle façon réaliser chaque tâche de façon à :
\begin{itemize}
	\item respecter les contraintes
	\item minimiser la durée totale de réalisation, $\Dmin$.
\end{itemize}

Modélisation : soit \GXA orienté, pondéré, défini par :
\begin{itemize}
	\item $X = T \cup \{\debut,\fin\}$
	\item $(x,y)\in A \Longleftrightarrow$ $x$ doit être réalisée avant $y$ et on pondère $(x,y)$ par $d_x$
	\item $(\debut,x)$ pondéré par 0
	\item $(x,\fin)$ pondéré par $d_x$
\end{itemize}

Exemple :
\begin{center}
\begin{tabular}{ccc}
 & Durée & Contraintes \\ 
\hline \hline
A & 2 & B,D avant A \\ 
\hline 
B & 7 & / \\ 
\hline 
C & 5 & B,D,E avant C \\ 
\hline 
D & 3 & / \\ 
\hline 
E & 6 & A avant E \\ 
\end{tabular} 
\end{center}

\newcommand{\ovun}[3]{\overset{{\color{blue}#2}}{\underset{{\color{red}#3}}{#1}}}

$\forall\mathcal{C}$ chemin de début à fin , $\Dmin \geqslant p(\mathcal{C})$.

$G$ est sans-circuit ssi le problème d'ordonnancement admet une solution. Dans ce cas, considérons un plch. de début à $x \neq$ début : le poids de ce chemin donne la date au plus tôt de début de réalisation de $x$. D'où :
\[ \Dmin = \text{poids d'un plch. de début à fin}\]

$\Rightarrow$ Algorithme de \textsc{Bellman} :

\begin{center}
\begin{tikzpicture}[scale = 1.5]

\tikzset{vertex/.style = {inner sep=2pt, draw = none}}
\tikzset{edge/.style = {->,> = latex'}}
% vertices
\node[vertex] (a) at  (0.7,1.4) {$\ovun{A}{7,B}{4}$};
\node[vertex] (b) at  (-0.7,1.4) {$\ovun{B}{0,\text{début}}{3}$};
\node[vertex] (c) at  (0.7,0) {$\ovun{C}{15,E}{6}$};
\node[vertex] (d) at  (-0.7,0) {$\ovun{D}{0,\text{début}}{2}$};
\node[vertex] (e) at  (0.7,0.7) {$\ovun{E}{9,A}{5}$};
\node[vertex] (deb) at  (-1.4,.7) {$\ovun{\text{début}}{0,\cdot}{1}$};
\node[vertex] (fin) at  (1.4,0.7) {$\ovun{\text{fin}}{20,C}{7}$};

%edges
\draw[edge] (deb) to node [above] {0} (b);
\draw[edge] (deb) to node [above] {0} (d);
\draw[edge] (b) to node [above] {7} (a);
\draw[edge] (b) to[bend left = 15] node [above] {7} (c);
\draw[edge] (d) to[bend right = 15] node [left] {3} (a);
\draw[edge] (d) to node [above] {3} (c);
\draw[edge] (a) to node [right] {2} (e);
\draw[edge] (e) to node [right] {6} (c);
\draw[edge] (c) to node [below] {5} (fin);

\end{tikzpicture}
\end{center}

\begin{flushright}
{\footnotesize \tred{Numérotation topologique} \\
{\color{blue} Marquage de \textsc{Bellman}}}
\end{flushright}

\begin{tikzpicture}[scale = 1]
\begin{axis}
[title = Diagramme de \textsc{Gantt}, height = 3cm, xlabel = temps, ylabel = ressources, width = 0.3\paperwidth, axis x line = bottom, axis y line =left, ymin = 0, ymax = 2.2, xmin = 0, xmax = 21]
\addplot+[blue, ybar interval, mark = none] coordinates {(0,1) (7,1) (9,1) (15,1) (20,1) }; 
\addplot+[blue, ybar interval, mark = none] plot coordinates {(0,2) (3,2)};
\addplot+[white, mark = none, very thick] plot coordinates {(3,0.01) (3,0.99)};
\node at (axis cs:3.5,0.5) [] {$B$};
\node at (axis cs:8,0.5) [] {$A$};
\node at (axis cs:12,0.5) [] {$E$};
\node at (axis cs:17.5,0.5) [] {$C$};
\node at (axis cs:1.5,1.5) [] {$D$};
\end{axis} 
\end{tikzpicture}

\section*{Parcours de graphes}

\subsection*{Parcours en largeur}

Soit \GXA et $r\in X$.

\begin{algorithm}[H]
Aucun sommet n'est examiné\;
Aucun sommet n'est marqué\;
Marquer $r$\;
Aucun arc n'est traversé\;
\While{$\exists$ un sommet $x$ marqué et un arc $(x,y)$ non traversé}
{
	\If{$x$ non examiné}{Examiner $x$}\;
	traverser $(x,y)$\;
	\If{$y$ non marqué}
	{
		marquer $y$\;
		père$(y) \longleftarrow x$\;
	}
}
\caption{\enquote{Moule} d'un parcours de graphe orientés en largeur}
\end{algorithm}

Gestion des objets \emph{marqués} selon une file :
\begin{itemize}
\item On ajoute les nouveaux sommets marqués à la fin de la file
\item On retire le sommet du début de la file une fois qu'il est examiné
\end{itemize}

\subsection*{Parcours en profondeur DFS}

Soit \GXA et $r\in X$.
\begin{flushright}
{\footnotesize
{\color{blue} cas préfixe} \\
{\color{red} cas postfixe}
}
\end{flushright}
\begin{algorithm}[H]
marquer $x$\;
{\color{blue} attribuer à $x$ le numéro préfixe courant}\;
\For{$(x,y)\in A$}
{
	traverser $(x,y)$\;
	\If{$y$ non marqué}
	{
		père($y$) $\longleftarrow x$\;
		appliquer DFS à $x$\;
	}
}
{\color{red} attribuer à $x$ le numéro postfixe courant}\;
\caption{DFS($x$)}
\end{algorithm}

\section*{Flux et flots}
\newcommand{\Sb}{\overline{S}}

Soit \GXA un graphe orienté. On dote chaque arc $a$ d'une capacité $c(a)$:
\[
\begin{array}{cc|ccl}
c & : & A & \to & \mathbb{R}^*_+  \\
 & & a & \mapsto & c(a) \\
\end{array}
\]
Soient $s$ et $p$ resp. sources et puits.

Un flot $f$ est défini par : \begin{enumerate}
\item $
\begin{array}{cc|ccl} 
f & : & A & \to & \mathbb{R}_+  \\
 & & a & \mapsto & f(a)\\
\end{array}
$
\item $\forall a \in A\quad 0 \leqslant f(a) \leqslant c(a) $
\item $\forall x \in X\backslash\{s,p\}$\\ 
 $\displaystyle\sum_{y\in V^-(x)}{f(x,y)} = \sum_{z\in V^+(x)}{f(x,z)}$
\end{enumerate}

On définit ainsi
\[v(f) = \displaystyle\sum_{z\in V^+(s)}{f(s,z)} - \sum_{y\in V^+(x)}{f(y,z)} \]

\subsection*{Problème de la coupe minimum}
Soit $G$ orienté, $c\, :\, A\, \rightarrow\, \mathbb{R}^*_+$, $s$ et $p$.

Une coupe $(S,\Sb)$ est une bipartition de $X$ avec :\begin{enumerate}
\item $S \cap \Sb = \varnothing$
\item $S \cup \Sb = X$
\item $s\in S$ et $p\in \Sb$
\end{enumerate}

On définit $c(S,\Sb) = \sum_{(x,y)\in S\times\Sb}{c(x,y)}$.

Soit $f$ un flot de $s$ à $p$. Soit $(S,\Sb)$ une coupe dans $G$. Alors :
\[ \displaystyle v(f) = \sum_{(x,y)\in S\times\Sb}{f(x,y)} - \sum_{(z,t)\in \Sb\times S}{f(z,t)} \]

\textbf{Corollaire 1 :} $\forall\,\text{flot}\,f, \,\forall\,\text{coupe}\,(S,\Sb)\\v(f) \leqslant c(S,\Sb)$.

\textbf{Corollaire 2 :} Soit $v_{\text{max}}$ la valeur maximale d'un flot et soit $c_{\text{min}}$ la capacité minimum d'une coupe. Alors $v_{\text{max}} \leqslant c_{\text{min}}$. De plus, s'il existe $f$ et $(S,\Sb)$ avec $v(f) = c(S,\Sb)$ alors $f$ est un flot maximum et $(S,\Sb)$ une coupe de capacité maximum.

Une \textit{chaîne augmentante} (cf. algorithme de marquage suivant) pour $f$ est une suite de sommets de la forme $x_1x_2\ldots x_k$ avec :
\begin{enumerate}
\item $\forall\, 1 \leqslant i \leqslant k\quad x_i \in X$
\item $\forall\, 1 \leqslant i \leqslant k-1$
	\begin{itemize}
	\item $(x_i,x_{i+1})\in A$ avec $f(x_i,x_{i+1}) < c(x_i,x_{i+1})$
	\item $(x_{i+1},x_i)\in A$ avec $f (x_{i+1},x_i) > 0$
	\end{itemize}
\end{enumerate}

On les représente de la sorte :
\begin{center}
\begin{tikzpicture}[scale = 1.5]

\tikzset{vertex/.style = {inner sep=2pt, draw = none}}
\tikzset{edge/.style = {->,> = latex'}}
% vertices
\node[vertex] (a) at  (-1,0) {$s$};
\node[vertex] (e) at  (0,0) {$b$};
\node[vertex] (b) at  (1,0) {$p$};

%edges
\draw[edge] (a) to node [above] {+$\varepsilon$} (e);
\draw[edge] (b) to node [above] {-$\varepsilon$} (e);

\end{tikzpicture}
\end{center}

\begin{algorithm}[H]
 $s$ est marqué par $(\Delta,+\infty)$ \;
 Les autres sommets sont non marqués \;
 Aucun sommet n'est examiné \;
 \While{$p$ non marqué \emph{et} $\exists$ un sommet marqué non-examiné}
 {
 	 Soit $x$ un tel sommet et $\alpha$ la val. absolue de la seconde composante de la marque de $x$ \;
 	 \For{$y\in V^+(x)$ non-marqué}
 	 {
 	 	\If{$f(x,y) < c(x,y)$}
 	 	{
 	 		marquer $y$ par \footnotesize{$\left(x,\min\left[\alpha,c(x,y)-f(x,y)\right]\right)$} \;
 	 	}
 	 }
 	 \For{$z\in V^-(x)$ non-marqué}
 	 {
 	 	\If{$f(x,y) > 0$}
 	 	{
 	 		marquer $z$ par \footnotesize{$\left(x,-\min\left[\alpha,f(z,x)\right]\right)$} \;
 	 	}
 	 }
 	 $x$ est examiné \;
 }
 
 \caption{Algorithme de \textsc{Ford} et \textsc{Fulkerson}}
\end{algorithm}

\subsection*{Théorème de Mayer}
Soit \GXA un graphe orienté.\\
\textbf{Problème 1 :} déterminer le nombre minimum d'\textit{arcs} à retirer $G$ pour laisser un graphe fortement connexe $=$ forte arc-connectivité.\\
\textbf{Problème 2 :} de même mais en enlevant des \textit{sommets} $=$ forte sommet-connectivité.\\
\textbf{Problème 3 \textit{\&} 4 :} mêmes problèmes avec $G$ non-orienté.

On résout le \textbf{problème 1} à l'aide des flots.

On définit $C_1$ et $C_2$ comme étant \emph{arcs-disjoints} s'il n'ont pas d'arêtes en commun.

\underline{\smash{Sous-problème :}} Soit $a$ et $b$ 2 sommets de $G$. Déterminer le minimum $N(a,b)$ d'arcs à retirer afin de plus avoir de chemin de $a$ vers $b$.\\
Posons $P(a,b)$ le nombre maximum de chemin arc-disjoint de $a$ vers $b$.

\smallskip
\textbf{Théorème :} En attribuant des capacités unitaires aux arcs de $G$, en considérant $a$ et $b$ resp. source et puit, et en considérant $f^*_{ab}$ le flot maximum de $a$ vers $b$, alors :
\[ N(a,b) = P(a,b) = v(f^*_{ab}) \]

Preuve par transitivité de l'inégalité :
\[ N(a,b) \overset{(1)}{\geqslant} P(a,b) \overset{(2)}{\geqslant} v(f^*_{ab}) \overset{(3)}{\geqslant} N(a,b)\]

La solution du \textbf{problème 1} est alors donnée par :
\[ \displaystyle \min_{a,b\in X}\left(N(a,b)\right) \]

On résout alors le \textbf{problème 2} en remplaçant les sommets de la sorte :
\begin{center}
\begin{tikzpicture}[scale = 1.5]

\tikzset{vertex/.style = {inner sep=2pt, draw = none}}
\tikzset{edge/.style = {->,> = latex'}}
% vertices
\node[vertex] (a) at  (-.7,0) {$a$};
\node[vertex] (b) at  (0,0) {$b$};
\node[vertex] (c) at  (0,0.7) {$c$};
\node[vertex] (d) at  (0.7,0) {$d$};

%edges
\draw[edge] (a) to (b);
\draw[edge] (b) to (d);

\draw[edge] (c)  to[bend left = 15] (b);
\draw[edge] (b) to[bend left = 15] (c);
\end{tikzpicture}\\
$\Downarrow$
\end{center}
\begin{center}
\begin{tikzpicture}[scale = 1.5]

\tikzset{vertex/.style = {inner sep=2pt, draw = none}}
\tikzset{edge/.style = {->,> = latex'}}
% vertices
\node[vertex] (a-) at  (-1.75,0) {$a^-$};
\node[vertex] (a+) at  (-1.05,0) {$a^+$};
\node[vertex] (b-) at  (-.35,0) {$b^-$};
\node[vertex] (b+) at  (.35,0) {$b^+$};
\node[vertex] (c-) at  (-.35,0.7) {$c^-$};
\node[vertex] (c+) at  (.35,0.7) {$c^+$};
\node[vertex] (d-) at  (1.05,0) {$d^-$};
\node[vertex] (d+) at  (1.75,0) {$d^+$};

%edges
\draw[edge] (a-) to (a+);
\draw[edge] (b-) to (b+);
\draw[edge] (c-) to (c+);
\draw[edge] (d-) to (d+);
\draw[edge] (a+) to (b-);
\draw[edge] (b+) to (d-);
\draw[edge] (b+) to (c-);
\draw[edge] (c+) to (b-);
\end{tikzpicture}
\end{center}
Puis on résout le problème 1 avec le graphe obtenu.

Le \textbf{problème 3} est résolu en remplaçant $\{x,y\}$ par $(x,y)$ et $(y,x)$. Attention, après la recherche d'arcs-disjoints il faut ré-ajuster les arcs afin d'y transformer en graphe non-orienté.

Le \textbf{problème 4} en résolu grâce au \textbf{problème 2} puis \textbf{1}.



\end{multicols*}
\end{document}
