\documentclass[a4paper, 8pt]{article}

\usepackage[utf8]{inputenc}
\usepackage[francais]{babel}
\usepackage[T1]{fontenc}
\usepackage{fancyhdr}
\usepackage{amsmath}
\usepackage{framed}
\usepackage[margin=0.7cm]{geometry}
\usepackage{varwidth}
\usepackage{multicol}
\usepackage{amsfonts}
\usepackage{amssymb}
\usepackage[babel=true]{csquotes}
\usepackage{titlesec}
 \usepackage[usenames,dvipsnames,pdf]{pstricks}
 \usepackage{epsfig}
 \usepackage{pst-grad} % For gradients
 \usepackage{pst-plot} % For axes
 \usepackage[space]{grffile} % For spaces in paths
 \usepackage{etoolbox} % For spaces in paths
\usepackage[vlined, boxruled]{algorithm2e}
\usepackage{pgf,tikz}
\usetikzlibrary{arrows}
\tikzset{
  font={\fontsize{7pt}{12}\selectfont}}
\usetikzlibrary{graphs}

\def\changemargin#1#2{\list{}{\rightmargin#2\leftmargin#1}\item[]}
\let\endchangemargin=\endlist 

\titleformat{\section}
 {\sffamily\Large\bfseries\raggedleft}{\thesection}{1em}{}[{\titlerule[1pt]}]
 
 
\titleformat{\subsection}
 {\sffamily\large\bfseries\centering}{\thesection}{0em}{}[{\titlerule[0pt]}]


\title{\sffamily\textbf{Fiche de révision \\ INF101} \thanks{Cours suivi avec Olivier \bsc{Hudry} }}
\author{Antonin \bsc{Godard}}
\date{}

\begin{document}

\begin{multicols*}{3}
\setlength{\parindent}{0pt}
\maketitle

\begin{changemargin}{0.5cm}{0.5cm} 
{\footnotesize Ce document est à lire après avoir suivi le cours. Le document fait donc l'hypothèse que le lecteur connaît au préalable les notions dont le document parle.}
\end{changemargin}


\textbf{Complexité:} mesure du temps nécessaire de calcul pour finir l'algorithme.\\ Soit $(P)$ un problème. Soit $I$ une instance de $(P)$. Soit $A$ un algorithme pour $(P)$.\\ Soit $f_A(I)$ le nombre d'opérations élémentaires effectuées pour compléter $I$. On définit alors $c_A$ par :
\[
\begin{array}{cc|ccl}
c_A & : & \mathbb{N} & \to & \mathbb{N} \\
 & & n & \mapsto & \max(f_A(I)) \\
\end{array}
\]

\section*{Structures de Données}

\textbf{Variables:} emplacement précis en mémoire. Possède une adresse et une taille variable.

\textbf{Tableau:} collection de variables. Contient des cases numérotées. Caractérisé par :\begin{itemize}
\item son adresse
\item sa taille
\item nature des entités conservées (de même type pour un tableau).
\end{itemize}
On accède à l'adresse via :\\ $ T\left[ i\right] = T\left[ 0\right] + i\times t $. ($t$ : taille).

\textbf{Listes chaînées :} suite de données homogènes (même type) auxquelles on accède de proche en proche, chaque donnée contenant l'adresse de l'élément suivant. Avantage : pas de problème de dimensionnement. Inconvénient : plus d'accès direct.

\textbf{Pile :} collection ordonnée de données respectant la stratégie \enquote{Last in, first out} (LIFO).

\textbf{File :} De même, \enquote{First in, first out} (FIFO).

\textbf{Arbres :} Un arbre $A$ est de la forme $A = (R, \underbrace{A_1, \ldots, A_k}_{\text{Arbres}})$. $R$ est la racine l'arbre $A$.\\
Terminologie : Soit $N$ un nœud de $A$.
\begin{itemize}
\item Les fils de $N$ sont les racines des sous-arbres de $N$;
\item Un nœud interne est un nœud admettant au moins 1 fils;
\item une feuille est un nœud n'ayant pas de fils;
\item tout nœud autre que la racine admet un père unique.
\end{itemize}

Profondeur de $N$ dans $A$ :
\[
\begin{array}{rc|l}
\text{prof}(N) & = & 0 \textit{ si } N=R  \\
 & & \text{prof}(\text{père de }N) \textit{ sinon.}  \\
\end{array}
\]

Hauteur de $A$ : $h = \displaystyle \max_{N\in A}\left(\text{prof}(N)\right).$

\textbf{Arbre complet :} Chaque nœud a deux fils.

\textbf{Arbre parfait :} $\forall i$, $(0 \leqslant i < h)$, le niveau $i$ est saturé.

\section*{Tri}

\textbf{Problème :} On dispose d'une collection de $n$ éléments, on souhaite les ordonner sous forme croissante.

\textbf{Tri insertion :} \`{A} la $i$\ieme étape, on considère les $i$ premier éléments triés, l'élément $i+1$ est alors inséré dans les éléments déjà triés par comparaison.
\begin{algorithm}[H]
 \KwData{$T$ tableau d'indices 1 à $n$}
 \For{$i$ variant de 2 à $n$}{
  $j \longleftarrow i$ \;
  $cle \longleftarrow T\left[j\right]$ \;
  \While{$j \geqslant 2$ et $T\left[j\right] > cle$}{
  	$T\left[j\right] \longleftarrow T\left[j-1\right]$ \;
  	$j \longleftarrow j-1$ \;
  }
  $T\left[j\right] \longleftarrow cle$ \;
 }
 \caption{Tri Insertion}
\end{algorithm}

\textbf{Tri sélection :} On cherche le plus petit élément, on le met à la 1\iere position; on cherche le 2\ieme plus petit, et etc$\hdots$

\begin{algorithm}[H]
 \KwData{$T$ tableau d'indices 1 à $n$}
 \For{$i$ variant de 1 à $n-1$}{
  $indicePetit \longleftarrow i$ \;
  $min \longleftarrow T\left[i\right]$ \;
  \For{$j$ variant de 1 à $n+1$}{
  \If{$T\left[j\right] < min$}{
  		$indicePetit \longleftarrow j$ \;
  		$min \longleftarrow T\left[j\right]$ \;
  	}
  	échanger($T$, $i$, $indicePetit$) \;
  }
 }
 \caption{Tri Sélection}
\end{algorithm}

\textbf{Tri rapide :} On utilise une fonction nommée $partition$ sur une liste qui sélectionne une donnée $cle$, et qui range les données plus petites que $cle$ à gauche de $cle$, puis range à droite de $cle$ les éléments plus grand que $cle$. Il est défini récursivement la plupart du temps.
\begin{algorithm}[H]
 \KwData{Données du tableau $T$ entre les indices $g$ et $d$}
  $cle \longleftarrow T\left[g\right]$ \;
  $i \longleftarrow g+1$ \;
  $j \longleftarrow d1$ \;
  \While{$j \geqslant i$}{
  	\While{$i \leqslant j$ et $T\left[i\right] \leqslant cle$}{$i \longleftarrow i+1$\;}
  	\While{$T\left[j\right] > cle$}{$j \longleftarrow j-1$\;}
  	\If{$i < j$}{
  		échanger($T$,$i$,$j$) \;
  		$i \longleftarrow i+1$\;
  		$j \longleftarrow j-1$\;
  	}
  }
  	échanger($T$,$g$,$j$)\;
  	\Return{j}
 \caption{partition($g$,$d$)}
\end{algorithm}
\begin{algorithm}[H]
 \KwData{$j$}
 \If{$g \leqslant d $}{
 	$j \longleftarrow$ partition($g$,$d$)\;
 	tri\_rapide($g$,$j-1$)\;
 	tri\_rapide($j+1$,$d$)\;
 }
 \caption{tri\_rapide($g$,$d$)}
\end{algorithm}

\begin{tikzpicture}
\node {root}
[level distance=7mm,
level 1/.style={sibling distance=20mm},
level 2/.style={sibling distance=10mm},
level 3/.style={sibling distance=5mm}]
child {node {left}
	child { node {child}}
	child {node {child}}
	}
child {node {right}
	child {node {child}}
	child {node {child}}
};
\end{tikzpicture}

\section*{Codage de \bsc{Huffman}}

\section*{Théorie des graphes}

\subsection*{Graphes Non-Orientés}
Un graphe non-orienté $G = (X,A)$ est un couple d'ensembles finis avec :
\begin{itemize}
\item $X = \left\{\text{sommets}\right\}$
\item $A = \left\{\mathbf{ar\hat{e}tes}\right\}$, les arêtes étant de la forme $\{x,y\}$ (le sens compte).
\item $m = \Omega(X)$ (cardinal)
\item $n = \Omega(Y)$
\item Pour une arête $a = \{x,y\}$ :
	\begin{itemize}
	\item $a$ est adjacente à $x$ et $y$
	\item $x$ et $y$ sont les extrémités de $a$
	\item $x$ et $y$ sont voisins
	\end{itemize}
\item Voisinage de $x$\\ $V(x) = \{y \in X / \{x,y\} \in A$
\item $d(x) = \Omega(V(x))$ et \begin{flushright}
$\sum_{x\in X}{d(x)} = 2m$
\end{flushright}
\item Chaîne : suite de sommets $x_1,x_2,\hdots,x_k$ avec :
\begin{enumerate}
\item $\forall \, 1\leqslant i \leqslant k \quad  x_i \in X$
\item $\forall \, 1\leqslant i \leqslant k-1 \quad \{x_i,x_{i+1}\} \in A$
\item $\forall \, 1< i< k \quad x_{i-1} \neq x_{i+1}$
\end{enumerate}
\item Chaîne simple (resp. élémentaire) : ne passe pas $2\times$ par une même arête (resp. sommet)
\item $x_1,x_2,\hdots,x_k$ élémentaire $\Longleftrightarrow \quad \forall \, 1\leqslant i,j \leqslant k \quad  x_i \neq x_j$
\item Cycle : suite de sommets $x_1,x_2,\hdots,x_k,(x_1)$ avec :
\begin{enumerate}
\item $\forall \, 1\leqslant i \leqslant k \quad  x_i \in X$
\item $\forall \, 1\leqslant i \leqslant k-1 \quad \{x_i,x_{i+1}\} \in A \,et\, \{x_k,x_{1}\} \in A$
\item $\forall \, 1< i< k \quad x_{i-1} \neq x_{i+1}$ \\$ et\, x_2 \neq x_k \,et\, x_1 \neq x_{k-1}$
\end{enumerate}
\item $G$ est dit connexe si $\forall\,x,y\in X \quad \exists\,\text{chaîne entre}\,x\,\text{et}\,y$. On définit de même les composantes connexes qui sont les sous-graphes connexes de $G$.
\item Une arête de $G$ est un \textit{isthme} si la suppression de cette arête fait croître le nombre de composantes connexes de $G$.
\item Un arbre est un graphe connexe sans cycle.
\end{itemize}

Exemple :
\begin{center}
\begin{tikzpicture}[scale = 1.5]

\tikzset{vertex/.style = {inner sep=2pt, draw = none}}
\tikzset{edge/.style = {-}}
% vertices

\node[vertex] (a) at  (0,.5) {$a$};
\node[vertex] (b) at  (-.5,0) {$b$};
\node[vertex] (c) at  (.5,1) {$c$};
\node[vertex] (d) at  (1,0) {$d$};
\node[vertex] (e) at  (1.7,1) {$e$};
\node[vertex] (f) at  (-1,1) {$f$};


%edges
\draw[edge] (a) to node[above]{7} (b);
\draw[edge] (a) to node[left]{3} (c);
\draw[edge] (f) to node[above]{4}(c);
\draw[edge] (e) to node[above]{0} (c);
\draw[edge] (d) to node[right]{-5} (e);
\draw[edge] (b) to node[below]{3} (d);
\draw[edge] (b) to node[left]{-1} (f);
\draw[edge] (d) to node[left]{-2} (c);
\end{tikzpicture}
\end{center}

Codage d'un graphe :

\begin{itemize}
	\item Matrice d'adjacence symétrique où a est la 1\iere colonne ou 1\iere ligne. Ici : \\
	 \[M = \left(\begin{smallmatrix}
	+\infty & 7 & 3 & +\infty & +\infty & +\infty \\ 
	 & +\infty & +\infty & 3 & +\infty & -1 \\ 
	 &  & +\infty & -2 & 0 & 4 \\ 
	 &  &  & +\infty & -5 & +\infty \\ 
	 & (sym) &  &  & +\infty & +\infty \\ 
	 &  &  &  &  & +\infty
	\end{smallmatrix} \right)\]
	
	\item Liste d'adjacence, où chaque ligne décrit les extrémités triées d'un élément avec leurs poids. Ici :
	\[\begin{smallmatrix}
	 a & \longrightarrow & b,7 & \longrightarrow & c,3 &  & &  &  \\ 
	 b & \longrightarrow & a,7 & \longrightarrow & d,3 & \longrightarrow & f,-1 &  &  \\ 
	 c & \longrightarrow & a,3 & \longrightarrow & d,-2 & \longrightarrow & e,0 & \longrightarrow & f,4 \\ 
	 d & \longrightarrow & b,3 & \longrightarrow & c,-2 & \longrightarrow & e,-5 &  &  \\ 
	 e & \longrightarrow & c,0 & \longrightarrow & d,-5 &  &  &  &  \\ 
	 f & \longrightarrow & b,-1 & \longrightarrow & c,4 &  &  &  & 
	 \end{smallmatrix}  \]
\end{itemize}
\textbf{Algorithme de \textsc{Krusgal} :}

\subsection*{Graphe Orientés}
Un graphe orienté $G = (X,A)$ est un couple d'ensembles finis avec :
\begin{itemize}
\item $X = \left\{\text{sommets}\right\}$
\item $A = \left\{\mathbf{arcs}\right\}$, les arcs étant de la forme $(x,y)$ (le sens compte).
\item $m = \Omega(X)$ (cardinal)
\item $n = \Omega(Y)$
\item Pour un arc $(x,y)$ :
	\begin{itemize}
	\item $x$ est prédécesseur de $y$
	\item $y$ est successeur de $x$
	\item $V^+(x) =$ voisinage extérieur de $x$
	\item $V^-(x) =$ voisinage intérieur de $x$
	\end{itemize}
\end{itemize}
\medskip
Exemple :
\begin{center}
\begin{tikzpicture}[scale = 1.5]

\tikzset{vertex/.style = {inner sep=2pt, draw = none}}
\tikzset{edge/.style = {->,> = latex'}}
% vertices
\node[vertex] (a) at  (0,2) {$A$};
\node[vertex] (e) at  (-1,1) {$E$};
\node[vertex] (b) at  (1,1) {$B$};
\node[vertex] (d) at  (-1,0) {$D$};
\node[vertex] (c) at  (1,0) {$C$};

%edges
\draw[edge] (a) to (e);
\draw[edge] (b) to (c);
\draw[edge] (c) to (d);

\draw[edge] (a)  to[bend left = 15] node [right] {abc} (b);
\draw[edge] (b) to[bend left = 15] (a);

\draw[edge] (b) to[bend left = 15] (e);
\draw[edge] (e) to[bend left = 15] (b);

\end{tikzpicture}
\end{center}

\section*{Flux et flots}

\subsection*{Théorème de Mayer}
Soit $G = (X,A)$ un graphe orienté.\\
\textbf{Problème 1 :} déterminer le nombre minimum d'\textit{arcs} à retirer $G$ pour laisser un graphe fortement connexe $=$ forte arc-connectivité.\\
\textbf{Problème 2 :} de même mais en enlevant des \textit{sommets} $=$ forte sommet-connectivité.\\
\textbf{Problème 3 \textit{\&} 4 :} mêmes problèmes avec $G$ non-orienté.

On résout le \textbf{problème 1} à l'aide des flots.

On définit $C_1$ et $C_2$ comme étant \emph{arcs-disjoints} s'il n'ont pas d'arêtes en commun.

\underline{\smash{Sous-problème :}}Soit $a$ et $b$ 2 sommets de $G$. Déterminer le minimum $N(a,b)$ d'arcs à retirer afin de plus avoir de chemin de $a$ vers $b$.\\
Posons $P(a,b)$ le nombre maximum de chemin arc-disjoint de $a$ vers $b$.

\smallskip
\textbf{Théorème :} En attribuant des capacités unitaires aux arcs de $G$, en considérant $a$ et $b$ resp. source et puit, et en considérant $f^*_{ab}$ le flot maximum de $a$ vers $b$, alors :
\[ N(a,b) = P(a,b) = v(f^*_{ab}) \]

Preuve par transitivité de l'inégalité :
\[ N(a,b) \overset{(1)}{\geqslant} P(a,b) \overset{(2)}{\geqslant} v(f^*_{ab}) \overset{(3)}{\geqslant} N(a,b)\]

La solution du \textbf{problème 1} est alors donnée par :
\[ \displaystyle \min_{a,b\in X}\left(N(a,b)\right) \]

On résout alors le \textbf{problème 2} en remplaçant les sommets de la sorte :
\begin{center}
\begin{tikzpicture}[scale = 1.5]

\tikzset{vertex/.style = {inner sep=2pt, draw = none}}
\tikzset{edge/.style = {->,> = latex'}}
% vertices
\node[vertex] (a) at  (-.7,0) {$a$};
\node[vertex] (b) at  (0,0) {$b$};
\node[vertex] (c) at  (0,0.7) {$c$};
\node[vertex] (d) at  (0.7,0) {$d$};

%edges
\draw[edge] (a) to (b);
\draw[edge] (b) to (d);

\draw[edge] (c)  to[bend left = 15] (b);
\draw[edge] (b) to[bend left = 15] (c);
\end{tikzpicture}\\
$\Downarrow$
\end{center}
\begin{center}
\begin{tikzpicture}[scale = 1.5]

\tikzset{vertex/.style = {inner sep=2pt, draw = none}}
\tikzset{edge/.style = {->,> = latex'}}
% vertices
\node[vertex] (a-) at  (-1.75,0) {$a^-$};
\node[vertex] (a+) at  (-1.05,0) {$a^+$};
\node[vertex] (b-) at  (-.35,0) {$b^-$};
\node[vertex] (b+) at  (.35,0) {$b^+$};
\node[vertex] (c-) at  (-.35,0.7) {$c^-$};
\node[vertex] (c+) at  (.35,0.7) {$c^+$};
\node[vertex] (d-) at  (1.05,0) {$d^-$};
\node[vertex] (d+) at  (1.75,0) {$d^+$};

%edges
\draw[edge] (a-) to (a+);
\draw[edge] (b-) to (b+);
\draw[edge] (c-) to (c+);
\draw[edge] (d-) to (d+);
\draw[edge] (a+) to (b-);
\draw[edge] (b+) to (d-);
\draw[edge] (b+) to (c-);
\draw[edge] (c+) to (b-);
\end{tikzpicture}
\end{center}
Puis on résout le problème 1 avec le graphe obtenu.

Le \textbf{problème 3} est résolu en remplaçant $\{x,y\}$ par$(x,y)$ et $(y,x)$. Attention, après la recherche d'arcs-disjoints il faut ré-ajuster les arcs afin d'y transformer en graphe non-orienté.

Le \textbf{problème 4} en résolu grâce au \textbf{problème 2} puis \textbf{1}.
\end{multicols*}
\end{document}
